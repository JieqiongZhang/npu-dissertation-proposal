%% File: npu-report-style.tex (西北工业大学研究生学位论文选题报告表)
%% Date: 2018/01/20
%% Author: Yang Zongze  (yangzongze@gmail.com)
%%
%% 使用说明
%% 1. 使用 `article' 类, 并使用 %% File: npu-report-style.tex (西北工业大学研究生学位论文选题报告表)
%% Date: 2018/01/20
%% Author: Yang Zongze  (yangzongze@gmail.com)
%%
%% 使用说明
%% 1. 使用 `article' 类, 并使用 %% File: npu-report-style.tex (西北工业大学研究生学位论文选题报告表)
%% Date: 2018/01/20
%% Author: Yang Zongze  (yangzongze@gmail.com)
%%
%% 使用说明
%% 1. 使用 `article' 类, 并使用 %% File: npu-report-style.tex (西北工业大学研究生学位论文选题报告表)
%% Date: 2018/01/20
%% Author: Yang Zongze  (yangzongze@gmail.com)
%%
%% 使用说明
%% 1. 使用 `article' 类, 并使用 \input{npu-report-style.tex} 加载该文件。
%% 2. 设置论文相关信息(参考 npu-report-sample.tex)
%%       \nputitle{-开题报告题目-}
%%       \npunumber{-学号-}
%%       \npuschool{-学院-}
%%       \npumajor{-专业-}
%%       \npuname{-姓名-}
%%       \npudegree{-学位-}
%%       \npusupervisor{-导师-}
%%       \nputype{-培养类别-}
%%       \npudate{-日期-}
%%       %% 设置 `论文类型': 取消或添加注释即可勾选相应类型
%%       \npucheckbasetrue       %  基础研究
%%       \npucheckapplytrue     %  应用研究
%%       % \npucheckenginetrue    %  工程技术
%%       % \npucheckoveralltrue   %  跨学科研究

\usepackage{xeCJK}
\usepackage{calc}
\usepackage[heading=true]{ctex}
\usepackage{lipsum}
\usepackage{tabularx}
\usepackage{multirow}
\usepackage{geometry}
\usepackage{fancybox}
\usepackage{filecontents}

\geometry{paperwidth=210mm,paperheight=297mm,%
  left=2.5cm,right=2.5cm,top=2.54cm,bottom=2.54cm}

\fancypage{%
  \setlength{\fboxsep}{\npusep}%
  \setlength{\shadowsize}{0pt}%
  \shadowbox}{}

\ifx\ctexset\undefined
  \CTEXsetup[format={\heiti\zihao{4}}]{section}
  \CTEXsetup[format={\heiti\zihao{-4}}]{subsection}
\else
  \ctexset{%
    section={%
      format={\heiti\zihao{4}},
      beforeskip=5pt,
      afterskip=5pt
    }}
  \ctexset{%
    subsection={%
      format={\heiti\zihao{-4}},
      beforeskip=0pt,
      afterskip=0pt
    }}
\fi

\let\npusection\section
\def\npustarsection#1{\npusection{#1}}
\def\section{\secdef\npusection\npustarsection}

\newlength{\npusep}
\newlength{\npuwidth}
\newlength{\npucorrect}
\newlength{\npucorrecttop}
\setlength{\npusep}{10pt}
\setlength{\npuwidth}{\textwidth}
\addtolength{\npuwidth}{2\npusep}
\setlength{\npucorrect}{5pt}
\addtolength{\npucorrect}{-\npusep}
\setlength{\npucorrecttop}{-10pt}
\addtolength{\npucorrecttop}{-\npusep}
\newlength{\npucone}
\setlength{\npucone}{2cm}
\newlength{\npuctwo}
\setlength{\npuctwo}{\npuwidth}
\addtolength{\npuctwo}{-\npucone}
\newcommand\npuitem[2][3cm]{\parbox[c][#1][c]{\npucone}{\centering #2}}
\newcommand\npusig[2][2.5cm]{\parbox[c][#1][c]{\npuctwo}{%
  \vskip 0pt plus15fil \hskip2cm #2:
  \hskip 0pt plus15fil 201\quad 年\quad 月\quad 日}}

\newif\ifnpucheckbase\npucheckbasefalse
\newif\ifnpucheckapply\npucheckapplyfalse
\newif\ifnpucheckengine\npucheckenginefalse
\newif\ifnpucheckoverall\npucheckoverallfalse

\def\npunew#1#2{%
  \expandafter\def\csname npuinner#1\endcsname{#2}
  \expandafter\def\csname npu#1\endcsname##1{%
    \expandafter\def\csname npuinner#1\endcsname{##1}}}
\npunew{title}{<开题报告题目>}
\npunew{number}{<学号>}
\npunew{school}{<学院>}
\npunew{major}{<专业>}
\npunew{name}{<姓名>}
\npunew{degree}{<学位>}
\npunew{supervisor}{<导师>}
\npunew{type}{<培养类别>}
\npunew{date}{<日期>}
\def\maketitlepage{
  \thisfancypage{}{}
  \begin{titlepage}
    \linespread{1.25}
    \begin{center}
      \newlength{\npunolenght}
      \newlength{\nputmplenght}
      \newcommand\npucu[1]{\uline{\hfill ##1 \hfill}}
      \newcommand\npumb[2][\nputmplenght]{\makebox[##1][s]{##2}}
      \newcommand\npucoveritem[2]{\npumb[\nputmplenght]{##1}\npucu{##2}\newline}
      \zihao{-4}
      \settowidth{\npunolenght}{20002010000000}
      \hfill 学号: \npumb[\npunolenght]{\npucu{\npuinnernumber}}
      \vskip 2cm
      {\zihao{1}\noindent \makebox[7cm][s]{西北工业大学}}\par
      \vskip 1.5cm
      {\kaishu\zihao{1}\noindent 研究生学位论文选题报告表}\par
      \kaishu\zihao{4}
      \vskip 0pt plus 5fil
      \settowidth{\nputmplenght}{学科、专业}
      \begin{minipage}{8.5cm}\linespread{1.8}\selectfont
        \noindent
        \npucoveritem{学院}{\npuinnerschool}
        \npucoveritem{学科、专业}{\npuinnermajor}
        \npucoveritem{姓名}{\npuinnername} 
        \npucoveritem{学位级别}{\npuinnerdegree}
        \npucoveritem{导师}{\npuinnersupervisor}
        \npucoveritem{培养类别}{\npuinnertype}
        \npucoveritem{报告日期}{\npuinnerdate}
      \end{minipage}
      \vskip 0pt plus 5fil\zihao{3}
      \makebox[3.5cm][s]{研究生院}
    \end{center}
  \end{titlepage}
  \clearpage}

\def\makeinnerpage{
  \thisfancypage{}{}
  \begin{center}
    \zihao{-2}  研究生学位论文选题报告的要求
  \end{center}\vskip1cm
  \newcounter{npulc}{%
  \zihao{-4}
  \begin{list}{\chinese{npulc}、}{%
    \settowidth\labelwidth{一、}\usecounter{npulc}}
  \item 硕士生的选题报告内容应包括文献综述、选题意义、研究内容、研究方案,论文
        工作量的估计、工作条件,预期达到的水平,存在的问题及拟采取的解决措施。
  \item 博士生的选题报告内容应包括文献综述、选题背景及其意义、研究内容、研究特
        色、工作难点、预期成果及其可能的创新点。
  \item 选题报告会应以学术活动的方式公开进行。
  \item 正式开题之前,研究生应在广泛阅读中、外文资料的基础上,深入了解拟选课题
        的国内外研究动态,把握所选课题的目的、意义和预期结果,明确课题工作的设
        想、方法和研究路径。
  \item 研究生在规定的时间内,写出选题报告初稿,经指导教师审阅同意后,由指导教
        师安排选题报告时间。选题报告未通过者,重新开题,若第二次选题报告仍通不
        过者,则按有关规定终止学籍。
  \item 选题报告不能按期完成者,应及时向研究生院培养办提出延期申请。
  \item 本表可以打印或用钢笔认真填写,若不够填写时,可另加附页。
  \end{list}}
  \clearpage}

\def\makehead{
  \hrule height0pt\vskip\npucorrecttop{%
  \renewcommand{\arraystretch}{1.5}
  \newcolumntype{C}{>{\hfil}X<{\hfil}}
  \noindent\hskip-\npusep
  \begin{tabularx}{\npuwidth}{>{\centering}X|C|C|C|C}
    论文题目 & \multicolumn{4}{c}{\npuinnertitle} \\ \hline
    \multirow{2}*{\parbox{0.18\npuwidth}{%
        \centering 论文类型\\(请在有关项目下作$\surd$记号)}} %
    & 基础研究 & 应用研究 & 工程技术 & 跨学科研究 \\ \cline{2-5}
    & \ifnpucheckbase $\surd$ \fi   & \ifnpucheckapply $\surd$ \fi  
    & \ifnpucheckengine $\surd$ \fi & \ifnpucheckoverall $\surd$\fi \\\hline
  \end{tabularx}}\vskip\npusep
  \noindent{\kaishu
    研究生在进行选题报告、听取意见后,整理成文(不少于3500字)。}
  \par}

\def\maketail{
  \filbreak
  \hrule height0pt \vfill
  {\noindent\hskip-\npusep\tabcolsep=0pt%
  \renewcommand{\arraystretch}{1.5}\zihao{-4}%
  \begin{tabularx}{\npuwidth}{c|X}\hline
  \npuitem[4.5cm]{指导教师\\意\qquad 见} & \npusig[4.5cm]{指导教师(签名)}  \\\hline
  \npuitem[3.5cm]{院\quad 系\\意\quad 见}& \npusig[3.5cm]{院系负责人(签名)}\\\hline
  \npuitem[]{研究生院\\复核时间}       & \npusig{复核人签名(签章)}\\\hline
  \npuitem[2.5cm]{备注}                  &  \\
  \end{tabularx}}
  \vskip\npucorrect}

\AtBeginDocument{\maketitlepage\makeinnerpage\makehead}
\AtEndDocument{\maketail}
\renewcommand{\normalsize}{\zihao{5}}
\pagestyle{plain}
%\linespread{1.5}
\endinput
 加载该文件。
%% 2. 设置论文相关信息(参考 npu-report-sample.tex)
%%       \nputitle{-开题报告题目-}
%%       \npunumber{-学号-}
%%       \npuschool{-学院-}
%%       \npumajor{-专业-}
%%       \npuname{-姓名-}
%%       \npudegree{-学位-}
%%       \npusupervisor{-导师-}
%%       \nputype{-培养类别-}
%%       \npudate{-日期-}
%%       %% 设置 `论文类型': 取消或添加注释即可勾选相应类型
%%       \npucheckbasetrue       %  基础研究
%%       \npucheckapplytrue     %  应用研究
%%       % \npucheckenginetrue    %  工程技术
%%       % \npucheckoveralltrue   %  跨学科研究

\usepackage{xeCJK}
\usepackage{calc}
\usepackage[heading=true]{ctex}
\usepackage{lipsum}
\usepackage{tabularx}
\usepackage{multirow}
\usepackage{geometry}
\usepackage{fancybox}
\usepackage{filecontents}

\geometry{paperwidth=210mm,paperheight=297mm,%
  left=2.5cm,right=2.5cm,top=2.54cm,bottom=2.54cm}

\fancypage{%
  \setlength{\fboxsep}{\npusep}%
  \setlength{\shadowsize}{0pt}%
  \shadowbox}{}

\ifx\ctexset\undefined
  \CTEXsetup[format={\heiti\zihao{4}}]{section}
  \CTEXsetup[format={\heiti\zihao{-4}}]{subsection}
\else
  \ctexset{%
    section={%
      format={\heiti\zihao{4}},
      beforeskip=5pt,
      afterskip=5pt
    }}
  \ctexset{%
    subsection={%
      format={\heiti\zihao{-4}},
      beforeskip=0pt,
      afterskip=0pt
    }}
\fi

\let\npusection\section
\def\npustarsection#1{\npusection{#1}}
\def\section{\secdef\npusection\npustarsection}

\newlength{\npusep}
\newlength{\npuwidth}
\newlength{\npucorrect}
\newlength{\npucorrecttop}
\setlength{\npusep}{10pt}
\setlength{\npuwidth}{\textwidth}
\addtolength{\npuwidth}{2\npusep}
\setlength{\npucorrect}{5pt}
\addtolength{\npucorrect}{-\npusep}
\setlength{\npucorrecttop}{-10pt}
\addtolength{\npucorrecttop}{-\npusep}
\newlength{\npucone}
\setlength{\npucone}{2cm}
\newlength{\npuctwo}
\setlength{\npuctwo}{\npuwidth}
\addtolength{\npuctwo}{-\npucone}
\newcommand\npuitem[2][3cm]{\parbox[c][#1][c]{\npucone}{\centering #2}}
\newcommand\npusig[2][2.5cm]{\parbox[c][#1][c]{\npuctwo}{%
  \vskip 0pt plus15fil \hskip2cm #2:
  \hskip 0pt plus15fil 201\quad 年\quad 月\quad 日}}

\newif\ifnpucheckbase\npucheckbasefalse
\newif\ifnpucheckapply\npucheckapplyfalse
\newif\ifnpucheckengine\npucheckenginefalse
\newif\ifnpucheckoverall\npucheckoverallfalse

\def\npunew#1#2{%
  \expandafter\def\csname npuinner#1\endcsname{#2}
  \expandafter\def\csname npu#1\endcsname##1{%
    \expandafter\def\csname npuinner#1\endcsname{##1}}}
\npunew{title}{<开题报告题目>}
\npunew{number}{<学号>}
\npunew{school}{<学院>}
\npunew{major}{<专业>}
\npunew{name}{<姓名>}
\npunew{degree}{<学位>}
\npunew{supervisor}{<导师>}
\npunew{type}{<培养类别>}
\npunew{date}{<日期>}
\def\maketitlepage{
  \thisfancypage{}{}
  \begin{titlepage}
    \linespread{1.25}
    \begin{center}
      \newlength{\npunolenght}
      \newlength{\nputmplenght}
      \newcommand\npucu[1]{\uline{\hfill ##1 \hfill}}
      \newcommand\npumb[2][\nputmplenght]{\makebox[##1][s]{##2}}
      \newcommand\npucoveritem[2]{\npumb[\nputmplenght]{##1}\npucu{##2}\newline}
      \zihao{-4}
      \settowidth{\npunolenght}{20002010000000}
      \hfill 学号: \npumb[\npunolenght]{\npucu{\npuinnernumber}}
      \vskip 2cm
      {\zihao{1}\noindent \makebox[7cm][s]{西北工业大学}}\par
      \vskip 1.5cm
      {\kaishu\zihao{1}\noindent 研究生学位论文选题报告表}\par
      \kaishu\zihao{4}
      \vskip 0pt plus 5fil
      \settowidth{\nputmplenght}{学科、专业}
      \begin{minipage}{8.5cm}\linespread{1.8}\selectfont
        \noindent
        \npucoveritem{学院}{\npuinnerschool}
        \npucoveritem{学科、专业}{\npuinnermajor}
        \npucoveritem{姓名}{\npuinnername} 
        \npucoveritem{学位级别}{\npuinnerdegree}
        \npucoveritem{导师}{\npuinnersupervisor}
        \npucoveritem{培养类别}{\npuinnertype}
        \npucoveritem{报告日期}{\npuinnerdate}
      \end{minipage}
      \vskip 0pt plus 5fil\zihao{3}
      \makebox[3.5cm][s]{研究生院}
    \end{center}
  \end{titlepage}
  \clearpage}

\def\makeinnerpage{
  \thisfancypage{}{}
  \begin{center}
    \zihao{-2}  研究生学位论文选题报告的要求
  \end{center}\vskip1cm
  \newcounter{npulc}{%
  \zihao{-4}
  \begin{list}{\chinese{npulc}、}{%
    \settowidth\labelwidth{一、}\usecounter{npulc}}
  \item 硕士生的选题报告内容应包括文献综述、选题意义、研究内容、研究方案,论文
        工作量的估计、工作条件,预期达到的水平,存在的问题及拟采取的解决措施。
  \item 博士生的选题报告内容应包括文献综述、选题背景及其意义、研究内容、研究特
        色、工作难点、预期成果及其可能的创新点。
  \item 选题报告会应以学术活动的方式公开进行。
  \item 正式开题之前,研究生应在广泛阅读中、外文资料的基础上,深入了解拟选课题
        的国内外研究动态,把握所选课题的目的、意义和预期结果,明确课题工作的设
        想、方法和研究路径。
  \item 研究生在规定的时间内,写出选题报告初稿,经指导教师审阅同意后,由指导教
        师安排选题报告时间。选题报告未通过者,重新开题,若第二次选题报告仍通不
        过者,则按有关规定终止学籍。
  \item 选题报告不能按期完成者,应及时向研究生院培养办提出延期申请。
  \item 本表可以打印或用钢笔认真填写,若不够填写时,可另加附页。
  \end{list}}
  \clearpage}

\def\makehead{
  \hrule height0pt\vskip\npucorrecttop{%
  \renewcommand{\arraystretch}{1.5}
  \newcolumntype{C}{>{\hfil}X<{\hfil}}
  \noindent\hskip-\npusep
  \begin{tabularx}{\npuwidth}{>{\centering}X|C|C|C|C}
    论文题目 & \multicolumn{4}{c}{\npuinnertitle} \\ \hline
    \multirow{2}*{\parbox{0.18\npuwidth}{%
        \centering 论文类型\\(请在有关项目下作$\surd$记号)}} %
    & 基础研究 & 应用研究 & 工程技术 & 跨学科研究 \\ \cline{2-5}
    & \ifnpucheckbase $\surd$ \fi   & \ifnpucheckapply $\surd$ \fi  
    & \ifnpucheckengine $\surd$ \fi & \ifnpucheckoverall $\surd$\fi \\\hline
  \end{tabularx}}\vskip\npusep
  \noindent{\kaishu
    研究生在进行选题报告、听取意见后,整理成文(不少于3500字)。}
  \par}

\def\maketail{
  \filbreak
  \hrule height0pt \vfill
  {\noindent\hskip-\npusep\tabcolsep=0pt%
  \renewcommand{\arraystretch}{1.5}\zihao{-4}%
  \begin{tabularx}{\npuwidth}{c|X}\hline
  \npuitem[4.5cm]{指导教师\\意\qquad 见} & \npusig[4.5cm]{指导教师(签名)}  \\\hline
  \npuitem[3.5cm]{院\quad 系\\意\quad 见}& \npusig[3.5cm]{院系负责人(签名)}\\\hline
  \npuitem[]{研究生院\\复核时间}       & \npusig{复核人签名(签章)}\\\hline
  \npuitem[2.5cm]{备注}                  &  \\
  \end{tabularx}}
  \vskip\npucorrect}

\AtBeginDocument{\maketitlepage\makeinnerpage\makehead}
\AtEndDocument{\maketail}
\renewcommand{\normalsize}{\zihao{5}}
\pagestyle{plain}
%\linespread{1.5}
\endinput
 加载该文件。
%% 2. 设置论文相关信息(参考 npu-report-sample.tex)
%%       \nputitle{-开题报告题目-}
%%       \npunumber{-学号-}
%%       \npuschool{-学院-}
%%       \npumajor{-专业-}
%%       \npuname{-姓名-}
%%       \npudegree{-学位-}
%%       \npusupervisor{-导师-}
%%       \nputype{-培养类别-}
%%       \npudate{-日期-}
%%       %% 设置 `论文类型': 取消或添加注释即可勾选相应类型
%%       \npucheckbasetrue       %  基础研究
%%       \npucheckapplytrue     %  应用研究
%%       % \npucheckenginetrue    %  工程技术
%%       % \npucheckoveralltrue   %  跨学科研究

\usepackage{xeCJK}
\usepackage{calc}
\usepackage[heading=true]{ctex}
\usepackage{lipsum}
\usepackage{tabularx}
\usepackage{multirow}
\usepackage{geometry}
\usepackage{fancybox}
\usepackage{filecontents}

\geometry{paperwidth=210mm,paperheight=297mm,%
  left=2.5cm,right=2.5cm,top=2.54cm,bottom=2.54cm}

\fancypage{%
  \setlength{\fboxsep}{\npusep}%
  \setlength{\shadowsize}{0pt}%
  \shadowbox}{}

\ifx\ctexset\undefined
  \CTEXsetup[format={\heiti\zihao{4}}]{section}
  \CTEXsetup[format={\heiti\zihao{-4}}]{subsection}
\else
  \ctexset{%
    section={%
      format={\heiti\zihao{4}},
      beforeskip=5pt,
      afterskip=5pt
    }}
  \ctexset{%
    subsection={%
      format={\heiti\zihao{-4}},
      beforeskip=0pt,
      afterskip=0pt
    }}
\fi

\let\npusection\section
\def\npustarsection#1{\npusection{#1}}
\def\section{\secdef\npusection\npustarsection}

\newlength{\npusep}
\newlength{\npuwidth}
\newlength{\npucorrect}
\newlength{\npucorrecttop}
\setlength{\npusep}{10pt}
\setlength{\npuwidth}{\textwidth}
\addtolength{\npuwidth}{2\npusep}
\setlength{\npucorrect}{5pt}
\addtolength{\npucorrect}{-\npusep}
\setlength{\npucorrecttop}{-10pt}
\addtolength{\npucorrecttop}{-\npusep}
\newlength{\npucone}
\setlength{\npucone}{2cm}
\newlength{\npuctwo}
\setlength{\npuctwo}{\npuwidth}
\addtolength{\npuctwo}{-\npucone}
\newcommand\npuitem[2][3cm]{\parbox[c][#1][c]{\npucone}{\centering #2}}
\newcommand\npusig[2][2.5cm]{\parbox[c][#1][c]{\npuctwo}{%
  \vskip 0pt plus15fil \hskip2cm #2:
  \hskip 0pt plus15fil 201\quad 年\quad 月\quad 日}}

\newif\ifnpucheckbase\npucheckbasefalse
\newif\ifnpucheckapply\npucheckapplyfalse
\newif\ifnpucheckengine\npucheckenginefalse
\newif\ifnpucheckoverall\npucheckoverallfalse

\def\npunew#1#2{%
  \expandafter\def\csname npuinner#1\endcsname{#2}
  \expandafter\def\csname npu#1\endcsname##1{%
    \expandafter\def\csname npuinner#1\endcsname{##1}}}
\npunew{title}{<开题报告题目>}
\npunew{number}{<学号>}
\npunew{school}{<学院>}
\npunew{major}{<专业>}
\npunew{name}{<姓名>}
\npunew{degree}{<学位>}
\npunew{supervisor}{<导师>}
\npunew{type}{<培养类别>}
\npunew{date}{<日期>}
\def\maketitlepage{
  \thisfancypage{}{}
  \begin{titlepage}
    \linespread{1.25}
    \begin{center}
      \newlength{\npunolenght}
      \newlength{\nputmplenght}
      \newcommand\npucu[1]{\uline{\hfill ##1 \hfill}}
      \newcommand\npumb[2][\nputmplenght]{\makebox[##1][s]{##2}}
      \newcommand\npucoveritem[2]{\npumb[\nputmplenght]{##1}\npucu{##2}\newline}
      \zihao{-4}
      \settowidth{\npunolenght}{20002010000000}
      \hfill 学号: \npumb[\npunolenght]{\npucu{\npuinnernumber}}
      \vskip 2cm
      {\zihao{1}\noindent \makebox[7cm][s]{西北工业大学}}\par
      \vskip 1.5cm
      {\kaishu\zihao{1}\noindent 研究生学位论文选题报告表}\par
      \kaishu\zihao{4}
      \vskip 0pt plus 5fil
      \settowidth{\nputmplenght}{学科、专业}
      \begin{minipage}{8.5cm}\linespread{1.8}\selectfont
        \noindent
        \npucoveritem{学院}{\npuinnerschool}
        \npucoveritem{学科、专业}{\npuinnermajor}
        \npucoveritem{姓名}{\npuinnername} 
        \npucoveritem{学位级别}{\npuinnerdegree}
        \npucoveritem{导师}{\npuinnersupervisor}
        \npucoveritem{培养类别}{\npuinnertype}
        \npucoveritem{报告日期}{\npuinnerdate}
      \end{minipage}
      \vskip 0pt plus 5fil\zihao{3}
      \makebox[3.5cm][s]{研究生院}
    \end{center}
  \end{titlepage}
  \clearpage}

\def\makeinnerpage{
  \thisfancypage{}{}
  \begin{center}
    \zihao{-2}  研究生学位论文选题报告的要求
  \end{center}\vskip1cm
  \newcounter{npulc}{%
  \zihao{-4}
  \begin{list}{\chinese{npulc}、}{%
    \settowidth\labelwidth{一、}\usecounter{npulc}}
  \item 硕士生的选题报告内容应包括文献综述、选题意义、研究内容、研究方案,论文
        工作量的估计、工作条件,预期达到的水平,存在的问题及拟采取的解决措施。
  \item 博士生的选题报告内容应包括文献综述、选题背景及其意义、研究内容、研究特
        色、工作难点、预期成果及其可能的创新点。
  \item 选题报告会应以学术活动的方式公开进行。
  \item 正式开题之前,研究生应在广泛阅读中、外文资料的基础上,深入了解拟选课题
        的国内外研究动态,把握所选课题的目的、意义和预期结果,明确课题工作的设
        想、方法和研究路径。
  \item 研究生在规定的时间内,写出选题报告初稿,经指导教师审阅同意后,由指导教
        师安排选题报告时间。选题报告未通过者,重新开题,若第二次选题报告仍通不
        过者,则按有关规定终止学籍。
  \item 选题报告不能按期完成者,应及时向研究生院培养办提出延期申请。
  \item 本表可以打印或用钢笔认真填写,若不够填写时,可另加附页。
  \end{list}}
  \clearpage}

\def\makehead{
  \hrule height0pt\vskip\npucorrecttop{%
  \renewcommand{\arraystretch}{1.5}
  \newcolumntype{C}{>{\hfil}X<{\hfil}}
  \noindent\hskip-\npusep
  \begin{tabularx}{\npuwidth}{>{\centering}X|C|C|C|C}
    论文题目 & \multicolumn{4}{c}{\npuinnertitle} \\ \hline
    \multirow{2}*{\parbox{0.18\npuwidth}{%
        \centering 论文类型\\(请在有关项目下作$\surd$记号)}} %
    & 基础研究 & 应用研究 & 工程技术 & 跨学科研究 \\ \cline{2-5}
    & \ifnpucheckbase $\surd$ \fi   & \ifnpucheckapply $\surd$ \fi  
    & \ifnpucheckengine $\surd$ \fi & \ifnpucheckoverall $\surd$\fi \\\hline
  \end{tabularx}}\vskip\npusep
  \noindent{\kaishu
    研究生在进行选题报告、听取意见后,整理成文(不少于3500字)。}
  \par}

\def\maketail{
  \filbreak
  \hrule height0pt \vfill
  {\noindent\hskip-\npusep\tabcolsep=0pt%
  \renewcommand{\arraystretch}{1.5}\zihao{-4}%
  \begin{tabularx}{\npuwidth}{c|X}\hline
  \npuitem[4.5cm]{指导教师\\意\qquad 见} & \npusig[4.5cm]{指导教师(签名)}  \\\hline
  \npuitem[3.5cm]{院\quad 系\\意\quad 见}& \npusig[3.5cm]{院系负责人(签名)}\\\hline
  \npuitem[]{研究生院\\复核时间}       & \npusig{复核人签名(签章)}\\\hline
  \npuitem[2.5cm]{备注}                  &  \\
  \end{tabularx}}
  \vskip\npucorrect}

\AtBeginDocument{\maketitlepage\makeinnerpage\makehead}
\AtEndDocument{\maketail}
\renewcommand{\normalsize}{\zihao{5}}
\pagestyle{plain}
%\linespread{1.5}
\endinput
 加载该文件。
%% 2. 设置论文相关信息(参考 npu-report-sample.tex)
%%       \nputitle{-开题报告题目-}
%%       \npunumber{-学号-}
%%       \npuschool{-学院-}
%%       \npumajor{-专业-}
%%       \npuname{-姓名-}
%%       \npudegree{-学位-}
%%       \npusupervisor{-导师-}
%%       \nputype{-培养类别-}
%%       \npudate{-日期-}
%%       %% 设置 `论文类型': 取消或添加注释即可勾选相应类型
%%       \npucheckbasetrue       %  基础研究
%%       \npucheckapplytrue     %  应用研究
%%       % \npucheckenginetrue    %  工程技术
%%       % \npucheckoveralltrue   %  跨学科研究

\usepackage{xeCJK}
\usepackage{calc}
\usepackage[heading=true]{ctex}
\usepackage{lipsum}
\usepackage{tabularx}
\usepackage{multirow}
\usepackage{geometry}
\usepackage{fancybox}
\usepackage{filecontents}

\geometry{paperwidth=210mm,paperheight=297mm,%
  left=2.5cm,right=2.5cm,top=2.54cm,bottom=2.54cm}

\fancypage{%
  \setlength{\fboxsep}{\npusep}%
  \setlength{\shadowsize}{0pt}%
  \shadowbox}{}

\ifx\ctexset\undefined
  \CTEXsetup[format={\heiti\zihao{4}}]{section}
  \CTEXsetup[format={\heiti\zihao{-4}}]{subsection}
\else
  \ctexset{%
    section={%
      format={\heiti\zihao{4}},
      beforeskip=5pt,
      afterskip=5pt
    }}
  \ctexset{%
    subsection={%
      format={\heiti\zihao{-4}},
      beforeskip=0pt,
      afterskip=0pt
    }}
\fi

\let\npusection\section
\def\npustarsection#1{\npusection{#1}}
\def\section{\secdef\npusection\npustarsection}

\newlength{\npusep}
\newlength{\npuwidth}
\newlength{\npucorrect}
\newlength{\npucorrecttop}
\setlength{\npusep}{10pt}
\setlength{\npuwidth}{\textwidth}
\addtolength{\npuwidth}{2\npusep}
\setlength{\npucorrect}{5pt}
\addtolength{\npucorrect}{-\npusep}
\setlength{\npucorrecttop}{-10pt}
\addtolength{\npucorrecttop}{-\npusep}
\newlength{\npucone}
\setlength{\npucone}{2cm}
\newlength{\npuctwo}
\setlength{\npuctwo}{\npuwidth}
\addtolength{\npuctwo}{-\npucone}
\newcommand\npuitem[2][3cm]{\parbox[c][#1][c]{\npucone}{\centering #2}}
\newcommand\npusig[2][2.5cm]{\parbox[c][#1][c]{\npuctwo}{%
  \vskip 0pt plus15fil \hskip2cm #2:
  \hskip 0pt plus15fil 201\quad 年\quad 月\quad 日}}

\newif\ifnpucheckbase\npucheckbasefalse
\newif\ifnpucheckapply\npucheckapplyfalse
\newif\ifnpucheckengine\npucheckenginefalse
\newif\ifnpucheckoverall\npucheckoverallfalse

\def\npunew#1#2{%
  \expandafter\def\csname npuinner#1\endcsname{#2}
  \expandafter\def\csname npu#1\endcsname##1{%
    \expandafter\def\csname npuinner#1\endcsname{##1}}}
\npunew{title}{<开题报告题目>}
\npunew{number}{<学号>}
\npunew{school}{<学院>}
\npunew{major}{<专业>}
\npunew{name}{<姓名>}
\npunew{degree}{<学位>}
\npunew{supervisor}{<导师>}
\npunew{type}{<培养类别>}
\npunew{date}{<日期>}
\def\maketitlepage{
  \thisfancypage{}{}
  \begin{titlepage}
    \linespread{1.25}
    \begin{center}
      \newlength{\npunolenght}
      \newlength{\nputmplenght}
      \newcommand\npucu[1]{\uline{\hfill ##1 \hfill}}
      \newcommand\npumb[2][\nputmplenght]{\makebox[##1][s]{##2}}
      \newcommand\npucoveritem[2]{\npumb[\nputmplenght]{##1}\npucu{##2}\newline}
      \zihao{-4}
      \settowidth{\npunolenght}{20002010000000}
      \hfill 学号: \npumb[\npunolenght]{\npucu{\npuinnernumber}}
      \vskip 2cm
      {\zihao{1}\noindent \makebox[7cm][s]{西北工业大学}}\par
      \vskip 1.5cm
      {\kaishu\zihao{1}\noindent 研究生学位论文选题报告表}\par
      \kaishu\zihao{4}
      \vskip 0pt plus 5fil
      \settowidth{\nputmplenght}{学科、专业}
      \begin{minipage}{8.5cm}\linespread{1.8}\selectfont
        \noindent
        \npucoveritem{学院}{\npuinnerschool}
        \npucoveritem{学科、专业}{\npuinnermajor}
        \npucoveritem{姓名}{\npuinnername} 
        \npucoveritem{学位级别}{\npuinnerdegree}
        \npucoveritem{导师}{\npuinnersupervisor}
        \npucoveritem{培养类别}{\npuinnertype}
        \npucoveritem{报告日期}{\npuinnerdate}
      \end{minipage}
      \vskip 0pt plus 5fil\zihao{3}
      \makebox[3.5cm][s]{研究生院}
    \end{center}
  \end{titlepage}
  \clearpage}

\def\makeinnerpage{
  \thisfancypage{}{}
  \begin{center}
    \zihao{-2}  研究生学位论文选题报告的要求
  \end{center}\vskip1cm
  \newcounter{npulc}{%
  \zihao{-4}
  \begin{list}{\chinese{npulc}、}{%
    \settowidth\labelwidth{一、}\usecounter{npulc}}
  \item 硕士生的选题报告内容应包括文献综述、选题意义、研究内容、研究方案,论文
        工作量的估计、工作条件,预期达到的水平,存在的问题及拟采取的解决措施。
  \item 博士生的选题报告内容应包括文献综述、选题背景及其意义、研究内容、研究特
        色、工作难点、预期成果及其可能的创新点。
  \item 选题报告会应以学术活动的方式公开进行。
  \item 正式开题之前,研究生应在广泛阅读中、外文资料的基础上,深入了解拟选课题
        的国内外研究动态,把握所选课题的目的、意义和预期结果,明确课题工作的设
        想、方法和研究路径。
  \item 研究生在规定的时间内,写出选题报告初稿,经指导教师审阅同意后,由指导教
        师安排选题报告时间。选题报告未通过者,重新开题,若第二次选题报告仍通不
        过者,则按有关规定终止学籍。
  \item 选题报告不能按期完成者,应及时向研究生院培养办提出延期申请。
  \item 本表可以打印或用钢笔认真填写,若不够填写时,可另加附页。
  \end{list}}
  \clearpage}

\def\makehead{
  \hrule height0pt\vskip\npucorrecttop{%
  \renewcommand{\arraystretch}{1.5}
  \newcolumntype{C}{>{\hfil}X<{\hfil}}
  \noindent\hskip-\npusep
  \begin{tabularx}{\npuwidth}{>{\centering}X|C|C|C|C}
    论文题目 & \multicolumn{4}{c}{\npuinnertitle} \\ \hline
    \multirow{2}*{\parbox{0.18\npuwidth}{%
        \centering 论文类型\\(请在有关项目下作$\surd$记号)}} %
    & 基础研究 & 应用研究 & 工程技术 & 跨学科研究 \\ \cline{2-5}
    & \ifnpucheckbase $\surd$ \fi   & \ifnpucheckapply $\surd$ \fi  
    & \ifnpucheckengine $\surd$ \fi & \ifnpucheckoverall $\surd$\fi \\\hline
  \end{tabularx}}\vskip\npusep
  \noindent{\kaishu
    研究生在进行选题报告、听取意见后,整理成文(不少于3500字)。}
  \par}

\def\maketail{
  \filbreak
  \hrule height0pt \vfill
  {\noindent\hskip-\npusep\tabcolsep=0pt%
  \renewcommand{\arraystretch}{1.5}\zihao{-4}%
  \begin{tabularx}{\npuwidth}{c|X}\hline
  \npuitem[4.5cm]{指导教师\\意\qquad 见} & \npusig[4.5cm]{指导教师(签名)}  \\\hline
  \npuitem[3.5cm]{院\quad 系\\意\quad 见}& \npusig[3.5cm]{院系负责人(签名)}\\\hline
  \npuitem[]{研究生院\\复核时间}       & \npusig{复核人签名(签章)}\\\hline
  \npuitem[2.5cm]{备注}                  &  \\
  \end{tabularx}}
  \vskip\npucorrect}

\AtBeginDocument{\maketitlepage\makeinnerpage\makehead}
\AtEndDocument{\maketail}
\renewcommand{\normalsize}{\zihao{5}}
\pagestyle{plain}
%\linespread{1.5}
\endinput
